% Vorlage Universität Tübingen
% erstellt von Ekaterina Kohler (ekaterina.kohler@googlemail.com)
% stark modifieziert und an die Bedürfnisse von QBiC angepasst von Nomi Meixner (nomi.meixner@student.uni-tuebingen.de)
\documentclass[a4paper,oneside,11pt,DIV=14]{scrartcl}
\usepackage{angebot-engl} %Uni-Design
\usepackage{eurosym} %Eurosymbol
\usepackage[german]{datenumber} %automatische Generierung des G�ltigkeitsdatums
%\usepackage{colortbl} % benötigt, falls man farbigen Tabellenhintergrund möchte
\usepackage{booktabs}% http://ctan.org/pkg/booktab


\newcommand{\farbe}{1} %Farbiges Uni-Logo? ja=1, nein=0
\newcommand{\fakult}{Medizinische Fakult\"{a}t T\"{u}bingen}	%Fakultät
\newcommand{\fbname}{Zentrum f\"{u}r Quantitative Biologie (QBiC)}	%wird aktuell nicht verwendet
\newcommand{\profname}{Dr. Stella Autenrieth} %Name
\newcommand{\funkt}{Gesch\"{a}ftsf\"{u}hrer()} %Funktion
\newcommand{\anschrift}{Auf der Morgenstelle 10\par 72076 T\"{u}bingen\par}
\newcommand{\tel}{Telefon +49-7071-29-72163\par}
\newcommand{\web}{\mbox{sven.nahnsen@uni-tuebingen.de}\par www.qbic.uni-tuebingen.de \par} % Beim Zeilenumbruch \par behalten.
\newcommand{\ang}{Angebot Nr.: QA2014016}	
\newcommand{\angbeschr}{FFPE Proteomics: Methodenetablierung und Experiment} %Bezeichnung des Angebots
\newcommand{\ort}{T\"{u}bingen}	%Ort
\newcommand{\datum}{\today}	%aktuelles Datum 
\newcommand{\empfaenger}{$institute \\ $PI \\ $street \\ $postalcode $city} % bis zu 5 Zeilen
\newcommand{\strasse}{$street} % für kleine Zeile über dem Adressfeld
\newcommand{\plzort}{$sender_postalcode $sender_city} % für kleine Zeile über dem Adressfeld%------------------------------------------------------
\begin{document}
\angebot %schreibt Header und Betreffzeile

\section*{Ablauf}

Dieses Angebot beinhaltet die Etablierungsarbeiten f�r die proteomeomische Analyse von Formalin-fixed, paraffin-embedded (FFPE) Gewebe. Desweiteren beinhaltet das Angebot die Kosten f�r proteomische Analysen f�r insgesamt 12 Proben (4 Herzen mit je 3 Bereiche). Im Vorfeld werden Metadaten zu den jeweiligen Proben ausgetauscht. Die Proben werden dann an das Labor zur massenspektrometrischen Messung durchgereicht. Das Angebot umfasst massenspektrometrische Analysen des Proteoms, sowie die bioinformatische Auswertung und eine Langzeitspeicherung der Rohdaten und Resultate.   
\section*{Preisangaben}
{
\renewcommand{\arraystretch}{2}
\footnotesize
\begin{tabular}{p{0.22\textwidth}p{0.22\textwidth}cccc}
 %\rowcolor{gold} %farbiger Hintergrund
 \hline
 \textbf{Arbeitspaket} & \textbf{Beschreibung} & \textbf{Anzahl} & \textbf{St�ckzahl (\euro)} & & \textbf{Gesamtpreis (\euro)} \\
 %\rowcolor{hellgold} %farbiger Hintergrund
 \hline
   Etablierungsarbeiten FFPE Gewebe Analyse & Etablierung der Probenvorbereitung (Testen von Protokollen), Etablierung der Datenakquisitionsmethode (Optimierung von Chromatographie und Massenspektrometrie), Etablierung der bioinformatischen Auswertestrategie und Qualit�tskontrolle   & 1 & -  & & 11.900\euro \\
      LC-MS/MS Analyse & Proteomische Analyse von FFPE Herzgewebe (inklusive Nachmessungen)  & 12 & 365,00  & & 4.380,00\euro \\
            Projekt- und Datenmanagement & Koordination; Bioinformatische Datenauswertung; Datenspeicherung  & 1 & -  & & 1.800,00\euro \\ %2 pipelines (500 EUR): 1 prozessierung, 1 x downstream analyse/qualit�tskontrolle, 100 EUR Data storage, 450 EUR Prozessierung der Samples; 1 x Projektmanagment (250 EUR)
 \hline
 &\multicolumn{5}{r}{\textbf{Voraussichtlicher Gesamtpreis}: \textbf{18.080,00\euro~(inkl. MWST)}}

\end{tabular}
}
\section*{Lieferfristen}
Die Lieferung der bioinformatischen Rohauswertung findet innerhalb 30 Tage nach Generierung der Daten statt. Genaue Kalenderwochenangaben k\"{o}nnen wir zum Zeitpunkt der Probenabgabe machen.
\section*{Vertrag}
\dateselectlanguage{german}
\setdatetoday
\addtocounter{datenumber}{21}%
\setdatebynumber{\thedatenumber}%
Bitte lassen Sie uns den unterschriebenen Vertrag zukommen, wenn Sie mit dem Angebot einverstanden sind. Das Angebot ist g\"{u}ltig bis \textbf{\datedate}.

% Ort/Datum, Unterschrift plus Linien oberhalb
\vspace{2.cm}
\renewcommand{\arraystretch}{2}
\noindent
\begin{tabular}{lp{2em}l}
 \hspace{6cm}   && \hspace{6cm} \\\cline{1-1}\cline{3-3}
 Ort, Datum     && Unterschrift
\end{tabular} 

\end{document}
