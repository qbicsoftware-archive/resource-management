% Vorlage Universität Tübingen
% erstellt von Ekaterina Kohler (ekaterina.kohler@googlemail.com)
% stark modifieziert und an die Bedürfnisse von QBiC angepasst von Nomi Meixner (nomi.meixner@student.uni-tuebingen.de)
\documentclass[a4paper,oneside,11pt,DIV=14]{scrartcl}
\usepackage{angebot-engl} %Uni-Design
\usepackage{eurosym} %Eurosymbol

%\usepackage{colortbl} % benötigt, falls man farbigen Tabellenhintergrund möchte
\usepackage{booktabs}% http://ctan.org/pkg/booktab

\newcommand{\farbe}{1} %Farbiges Uni-Logo? ja=1, nein=0

\newcommand{\fakult}{$sender_title \par $sender_faculty \par}
\newcommand{\profname}{$sender_function} %Name
\newcommand{\funkt}{$sender_name} %Funktion
\newcommand{\anschrift}{$sender_institute \par $sender_street \par $sender_postalcode $sender_city \par}
\newcommand{\tel}{Ansprechpartner \par $sender_name \par Telefon $sender_phone}
\newcommand{\web}{\mbox{$sender_email}\par $sender_url \par} % Beim Zeilenumbruch \par behalten.

\newcommand{\ang}{Rechnung}
%\newcommand{\ang}{Rechnungs-Nr:}{$invoice_number}
%\newcommand{\ang}{Anfordernde Kostenstelle:}{$project_number}
\newcommand{\angbeschr}{$project_short_description} %Bezeichnung des Angebots
\newcommand{\ort}{$sender_city} %T\"{u}bingen}%Ort
\newcommand{\datum}{\today}	%aktuelles Datum
\newcommand{\empfaenger}{$PI \\ $institute \\ $street \\ $postalcode $city} % bis zu 5 Zeilen
\newcommand{\strasse}{$sender_street} % für kleine Zeile über dem Adressfeld
\newcommand{\plzort}{$sender_postalcode $sender_city} % für kleine Zeile über dem Adressfeld%------------------------------------------------------
\begin{document}
\angebot %schreibt Header und Betreffzeile

\section*{Ablauf}
$project_description
\section*{Preisangaben}
{
\renewcommand{\arraystretch}{2}
\footnotesize
\begin{tabular}{ccp{0.6\textwidth}c}
 %\rowcolor{gold} %farbiger Hintergrund
 \hline
 \textbf{Datum} & \textbf{Menge} & \textbf{Bezeichnung} & \textbf{Gesamt (\euro)} \\
 %\rowcolor{hellgold} %farbiger Hintergrund
 #foreach($row in $costs)
   \hline
   $row.date & $row.time_frame & $row.description & $row.cost \\
 #end
 \hline \hline
 &\multicolumn{3}{r}{{Summe Netto}: {$total_cost \euro}}
  & &\multicolumn{3}{r}{{MwSt. (0\%)}: {$total_cost \euro}}
   & &\multicolumn{3}{r}{\textbf{Bruttosumme}: \textbf{$total_cost \euro}} \\ 
   \hline

\end{tabular}
}
\vspace{10 mm}

\textbf{Bitte Rechnungsbetrag nicht anweisen. Der betrag wird in den n\"{a}chsten
Tagen der von Ihnen angegebenen Kostenstelle abgebucht.}

\vspace{10 mm}
Mit freundlichen Gr\"{u}{\ss}en
\end{document}
